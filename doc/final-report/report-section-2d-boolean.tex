For clipping the resulting polygon set with a bounding polygon, we utilized CGAL's boolean set operations.
These operations allow obtaining the set union, intersection, difference etc. on polygons (optionally
with holes) or sets thereof.

The intersection-operation was of particular interest to us. It comes in several flavors:
\begin{itemize}
\item A global function ``intersection(InputIterator, InputIterator, OutputIterator)'' that iterates over the
  polygons (with holes) in the given range and writes their aggregate intersection to the output stream.
  The ``OutputIterator'' is used to append the list of resulting polygons to that stream.
  This function was of no use to us, because our input does not necessarily consist of single polygons,
  but rather of polygon sets ("multipolygons").
  
\item Another global function
  ``intersection(InputIterator1, InputIterator1, InputIterator2, InputIterator2, OutputIterator)'' that
  takes two ranges of polygons (with holes), interprets them as multipolygons and writes their intersection
  (which is, again, also a multipolygon) to the output stream referenced by the ``OutputIterator''.
  
\item There are several overloaded versions of ``intersection()'' that take ``Polygon\_2''s or ``Polygon\_with\_holes\_2''s
  directly and write their intersection to an ``OutputIterator'' like above.
  
\item Also, there is a class ``Polygon\_set\_2'' that encapsulates a list of polygons (with holes), using CGAL
  Arrangements. This class also has several ``intersection()''-methods defined, especially one that takes
  another ``Polygon\_set\_2'' as an argument and writes the two polygon sets' intersection back into the
  receiver.
\end{itemize}
  
We decided to use the last option, because it makes the code simpler and cleaner. The following subsections
will give a brief overview of the classes ``Polygon\_2'', ``Polygon\_with\_holes\_2'' and ``Polygon\_set\_2'', which
we used in the process.

\subsubsection{Polygon\_2}

This is mostly a wrapper for a list of ``Point\_2''-instances. It can, for example, be built like this:

\begin{lstlisting}[numbers=none]
    Polygon_2 p;
    p.push_back(Point(0, 0));
    p.push_back(Point(0, 1));
    p.push_back(Point(1, 0));
\end{lstlisting}

It also supports query methods and methods for modification. For example:

\begin{lstlisting}[numbers=none]
    if (p.orientation() != CGAL::COUNTERCLOCKWISE) p.reverse_orientation();
\end{lstlisting}

Also, one can iterate the contained vertices using the methods ``vertices\_begin()'' and ``vertices\_end()''.
Other than that, we do not touch ``Polygon\_2''-instances directly.

\subsubsection{Polygon\_with\_holes\_2}

This contains an (optional) boundary polygon (which has to have a counter-clockwise orientation) along with a list
of hole polygons (all of which need to have a clockwise orientation). The contained polygons are all of type
``Polygon\_2''. Instances can be created by providing the boundary as well as the list of holes (though an iterator
range) to the constructor:

\begin{lstlisting}[numbers=none]
    std::list<Polygon> holes;
    Polygon_2 hole;
    hole.push_back(Point(0.1, 0.1));
    hole.push_back(Point(0.1, 0.9));
    hole.push_back(Point(0.9, 0.1));
    holes.push_back(hole)
    
    Polygon_with_holes_2 ph(p, holes.begin(), holes.end());
\end{lstlisting}

The boundary can be accessed with:

\begin{lstlisting}[numbers=none]
    Polygon boundary = ph.boundary();
\end{lstlisting}

Likewise, the list of holes can be accessed through the ``holes\_begin()'' and ``holes\_end()'' iterators.

\subsubsection{Polygon\_set\_2}

A ``Polygon\_set\_2'' is a collection of several ``Polygon\_with\_holes\_2''s and is therefore, in our opinion,
the most natural way to use CGAL's boolean set operations. After all, the result is always a collection of potentially
several polygons, and also our inputs have that form.

It is, however, a bit tricky to use. There are several preconditions for the ``Polygon\_with\_holes\_2''s to be added
to a set:

\begin{itemize}
\item All ``Polygon\_2''s contained must be simple polygons.
\item Their boundaries must be closed. (When using ``push\_back()'' to construct them, this seems to be taken care of
  automatically. However, the first and last vertex must not be equal.)
\item They must have a correct orientation. The boundary must be oriented counter-clockwise, while all holes have to
  be oriented clockwise.
\item The holes must lie completely within the boundary's area (except potentially on single vertices).
\end{itemize}

According to documentation \cite{cgal:fwzh-rbso2-12b}, these rules define ``valid'' polygons (with holes) in the context of boolean set
operations. (However, we find the documentation to be a bit vague in this regard.)
If any of these conditions are not met, the set operations, including ``join''s and initial construction from a
``Polygon\_with\_holes\_2'' will simply segfault without mercy. This is also true for the other versions of
``intersection'' introduced earlier.

We use two variants of building a ``Polygon\_set\_2''. The first one is simply using the default constructor
to create an empty polygon set:

\begin{lstlisting}[numbers=none]
    Polygon_set_2 s1;
\end{lstlisting}

The second version initializes the created set to contain a first ``Polygon\_with\_holes\_2'':

\begin{lstlisting}[numbers=none]
    Polygon_set_2 s2(ph);
\end{lstlisting}
    
Once created, these objects may be used with boolean set operations:

\begin{lstlisting}[numbers=none]
    s1.join(s2);
\end{lstlisting}
    
When finished, the resulting collection of polygons (with holes) can be extracted like this:

\begin{lstlisting}[numbers=none]
    std::list<Polygon_with_holes_2> polygons;
    s1.polygons_with_holes(std::back_inserter(polygons));
\end{lstlisting}
    
This writes the contained list of polygons to the specified collection for further processing.
