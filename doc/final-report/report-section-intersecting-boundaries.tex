The purpose of this part of the project is to clip the resulting geometry, represented as a
polygon (with holes) or a set thereof, to a bounding region, also defined by a polygon.
Both resulting geometry and bounding polygon are available as serialized strings that adhere to
a subset of the ``well-known-text'' format.

Therefore, the task is to parse the polygonal strings into instances of CGAL's ``Polygon\_set\_2''
class, apply the ``intersection'' boolean set operation and then write the resulting polygon set
out in that same format.

WKT is a simple clear-text format. The subset used in this project defines expressions for polygons
(``POLYGON'') or polygon sets (``MULTIPOLYGON''), with optional holes.
The following example defines a simple polygon with one hole:

\texttt{POLYGON ((0 0, 0 1, 1 1, 1 0, 0 0), (0.2 0.2, 0.2 0.8, 0.8 0.8, 0.2 0.2))}

Each polygon definition contains a mandatory boundary (the first list of points) as well as an
arbitrary amount of holes (one in this case).

The following sections will give a brief overview of the components written for that purpose.

\subsubsection{Tokenizer}

We first tried to use the boost-Lexer ``Spirit'', however this seemed like overkill for a language as small
as this, so we implemented a simple hand-rolled solution.

The class ``TokenStream'' takes the input stream and breaks it up into several instances of ``Token''.
Among others, it provides the following methods:

\begin{itemize}
\item \texttt{Token peek()}: returns the next token on the input stream.
\item \texttt{Token consume(Token::Type type)}: Removes the ``Token'' of type ``type'' from the front of the stream, or
  throws an exception if it does not match.
\end{itemize}
  
``Token''s simply contain an ``enum'' value of ``Token::Type'' that defines the token type as well as a
``std::string'' containing the matched text. Token types include parentheses, comma, numbers and the
recognized keywords ``POLYGON'', ``MULTIPOLYGON'' and ``EMPTY''.

This is not a full-featured lexer, it only contains the features we need.

\subsubsection{Parser}

``Parser'' is a simple recursive descent parser that operates on a ``TokenStream'' passed in through the constructor.
The parsing contexts are public methods that query the token stream and recursively build up the required CGAL-objects
from the found tokens.
The following method, for example, parses a and builds up a CGAL ``Polygon\_set\_2'':

\begin{lstlisting}[numbers=none]
    Polygon_set Parser::parsePolygonSet() {
        _tokens.consume(Token::LEFT_PARENTHESIS);
        Polygon_set result(parsePolygonWithHoles());
        while (_tokens.peek().type() == Token::COMMA) {
            _tokens.consume(Token::COMMA);
            result.join(parsePolygonWithHoles());
        }
        _tokens.consume(Token::RIGHT_PARENTHESIS);
        return result;
    }
\end{lstlisting}
    
It in turn calls the ``parsePolygonWithHoles()'' method that parses and returns ``Polygon\_with\_holes\_2''s.

There is an inherent mismatch between the WKT-represantation and the CGAL-represantation of the used objects.
For example, WKT does not specify the polygons' orientations, while CGAL requires a specific orientation.
Also, WKT mandates the closing of the polygon vertex set to be made explicit by adding the first point again as
a last point; in CGAL, this is interpreted as a non-simple polygon and crashes the construction of
``Polygon\_set\_2''s with a segfault. Things like these are handled by the ``Parser''.

The root parsing context is ``Polygon\_set parsePolygonSetDefinition()'', which is called by the main program
to initiate parsing.

\subsubsection{Main program}

This glues the components together and calls CGAL's ``intersection'' functionality.
It takes (at least) one polygon filename, which is read in to construct the initial ``Polygon\_set''. It then
iterates over the remaining passed filenames, and for each of them parses the contents and aggregates the intersection
into the original polygon set. Afterwards, the polygon set containing the aggregate intersection is written to
standard out in WKT-format.

The program has the following syntax:

\begin{lstlisting}[numbers=none]
./intersect <polygonFile>, ...
\end{lstlisting}
