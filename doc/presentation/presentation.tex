\documentclass{beamer}

\usepackage[ngerman]{babel}
\usepackage[utf8]{inputenc}
\usepackage{listings}

\lstset{
    language=c++,
    basicstyle=\tiny,
    showspaces=false,
    showstringspaces=false
}
\setbeamertemplate{footline}{\insertframenumber/\inserttotalframenumber}

\title{Generierung von Apollonius-Diagrammen auf Basis von ortsbezogenen Daten}

\author[Sebastian Kürten, Philipp Borgers, Lukas Maischak]{
    Sebastian Kürten
    \newline Lukas Maischak
    \newline Philipp Borgers
}

\begin{document}

\begin{frame}
    \titlepage
\end{frame}

\begin{frame}
    \frametitle{Thema}
    \begin{itemize}
        \item Approximation von Gemeindegrenzen mit Hilfe von Apollonius-Diagrammen
        (Additiv gewichtete Voronoi-Diagramme)
        \item Datenbasis: sogenannte Ortsknoten aus der Openstreetmap Datenbank
        \item Eingabe: Menge von Knoten mit Metadaten (Einwohnerzahl)
        \item Ausgabe: Zu jedem Knoten konstruiere ein Polygon, welches die der Gemeinde
        zugeordnete Fläche bestimmt.
        \item Dabei sollen die Polygone eine Zerlegung der Ebene bilden.
    \end{itemize}
\end{frame}

\begin{frame}
    \frametitle{Voronoi-Diagramme}
    \begin{itemize}
        \item Was ist das überhaupt?
        \item Definition?
        \item Beispiel-Bild?
    \end{itemize}
\end{frame}

\begin{frame}
    \frametitle{Apollonius Diagramme (Gewichtete Voronoi-Diagrame)}
    \begin{itemize}
        \item Unterschied zum Voronoi-Diagramm
        \item Definition
        \item Beispiel
    \end{itemize}
\end{frame}

\begin{frame}
    \frametitle{CGAL Implementierung}
    \begin{itemize}
        \item Code zur Generierung eines Graphen?
    \end{itemize}
\end{frame}

\begin{frame}
    \frametitle{CGAL Implementierung - Dual Graph}
    \begin{itemize}
        \item Code zur Generierung eines Graphen?
    \end{itemize}
\end{frame}

\begin{frame}
    \frametitle{Vorverarbeitung}
    \begin{itemize}
        \item Aus den OpenStreetMap-Daten werden Place-Nodes (city, town, village) extrahiert
        \item Zu jedem Place-Node halten wir die OSM-Id und den Koordinaten fest
        \item Format: <Openstreetmap Node Id> <Longitude> <Latitude> <type>
    \end{itemize}
\end{frame}

\begin{frame}
    \frametitle{Ausgabe}
    \begin{itemize}
        \item foo
    \end{itemize}
\end{frame}

\begin{frame}
    \frametitle{Demo}
    \begin{center}
        Ludi incipiant
    \end{center}
\end{frame}

\begin{frame}
    \frametitle{Fragen? Danke}
    \begin{center}
    Fragen?
    \newline Vielen Dank für Ihre Geduld!
    \end{center}
\end{frame}

\end{document}
