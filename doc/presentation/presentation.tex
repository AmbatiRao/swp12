\documentclass{beamer}

\usepackage[ngerman]{babel}
\usepackage[utf8]{inputenc}
\usepackage{listings}
% this makes < > work
\usepackage[T1]{fontenc}
\usepackage{graphicx}
\usepackage{caption}
\usepackage{subcaption}

\lstset{
    language=c++,
    basicstyle=\tiny,
    showspaces=false,
    showstringspaces=false
}
\setbeamertemplate{footline}{\insertframenumber/\inserttotalframenumber}

\title{Using CGAL to create Apollonius diagrams based on OpenStreetMap data}

\author[Sebastian Kürten, Philipp Borgers, Lukas Maischak]{
    Philipp Borgers
    \newline Sebastian Kürten
    \newline Lukas Maischak
}

\begin{document}

\begin{frame}
    \titlepage
\end{frame}

\begin{frame}
    \frametitle{Thema}
    \begin{itemize}
        \item Approximation von Gemeindegrenzen mit Hilfe von Apollonius-Diagrammen
        (Additiv gewichtete Voronoi-Diagramme)
        \item Datenbasis: sogenannte Ortsknoten aus der Openstreetmap Datenbank
        \item Eingabe: Menge von Knoten mit Metadaten (Einwohnerzahl)
        \item Ausgabe: Zu jedem Knoten konstruiere ein Polygon, welches die der Gemeinde
        zugeordnete Fläche bestimmt.
        \item Dabei sollen die Polygone eine Zerlegung der Ebene bilden.
    \end{itemize}
\end{frame}

\begin{frame}
    \frametitle{Apollonius diagram (2D)}
    \begin{itemize}
        \item aka. additively weighted Voronoi diagram
        \item Subdivision of the plane into connected regions (cells)
        \item Defined over a set of sites $P_i = (c_i, w_i)$
        \item A cell contains all points that are closer to $P_i$ than to any
            other site $P_j$
    \end{itemize}
\end{frame}

\begin{frame}
    \frametitle{Apollonius graph}
    \begin{itemize}
        \item The dual of the Apollonius diagram
        \item Nodes of the graph correspond to cells in the diagram.
        \item An arc in the graph corresponds to an edge of a cell in the diagram.
        \item If two cells of the diagram are connected, there is an arc between
        the corresponding nodes of the graph.

    \end{itemize}
\end{frame}

\begin{frame}
    \frametitle{Examples in 2D}
    \begin{figure}[htp]
            \centering
            \begin{subfigure}[b]{0.48\textwidth}
                    \centering
                    \includegraphics[width=\textwidth]{../final-report/images/apollonius-graph.png}
                    \caption{Apollonius graph}
                    \label{fig:apo-graph}
            \end{subfigure}
            \begin{subfigure}[b]{0.48\textwidth}
                    \centering
                    \includegraphics[width=\textwidth]{../final-report/images/apollonius-diagram.png}
                    \caption{Apollonius diagram}
                    \label{fig:apo-diagram}
            \end{subfigure}
            \caption{Apollonius graph and Apollonius diagram}\label{fig:apollonius}
    \end{figure}
\end{frame}

\begin{frame}
    \frametitle{CGAL Apollonius graph}
    \begin{itemize}
        \item CGAL implements the Apollonius graph
        \item Access to sites, vertices, faces edges of the graph
        \item Use the dual function to create the Apollonius diagram
    \end{itemize}
\end{frame}

\begin{frame}
    \frametitle{Example Apollonius graph in CGAL}
    \lstinputlisting{../final-report/code/init-apollonius-graph.c}
\end{frame}

\begin{frame}
    \frametitle{Access the dual in CGAL}
    \lstinputlisting{../final-report/code/access-dual.c}
\end{frame}

\begin{frame}
    \frametitle{Preprocessing}
    \begin{itemize}
        \item Extract place nodes (city, town, village) from OpenStreetMap dataset
        \item Use the position information and the osm-id for later reference
        \item Format: <Openstreetmap Node Id> <Longitude> <Latitude> <type>
        \item Written in JAVA
    \end{itemize}
\end{frame}

\begin{frame}
    \frametitle{Preprocessing: Input data (OSM)}
    \lstinputlisting{../final-report/data/place-nodes.osm}
\end{frame}

\begin{frame}
    \frametitle{Preprocessing: Output data}
    \lstinputlisting{../final-report/data/place-nodes.txt}
\end{frame}

\begin{frame}
    \frametitle{Generate apollonius graph}
    \begin{itemize}
        \item Insert weighted points (sites) into a apollonius graph instance
        \item Insert artificial sites so we do not create infinite rays
    \end{itemize}
\end{frame}

\begin{frame}
    \frametitle{Generate cells (polygons)}
    \begin{itemize}
        \item Starting from the incident edges of a vertix
        \item Different types of objects involed
        \item Line, Segment, Ray, Hyperbola, Hyperbola\_segment, Hyperbola\_ray
        \item implemented in the handleDual function
    \end{itemize}
\end{frame}

\begin{frame}
    \frametitle{Output-Formats: WKT}
    \lstinputlisting{../final-report/code/output.wkt}
\end{frame}

\begin{frame}
    \frametitle{Output-Formats: WKT}
    \lstinputlisting{../final-report/code/wkt.c}
\end{frame}

\begin{frame}
    \frametitle{Output-Formats: GeoJSON}
    \lstinputlisting{../final-report/code/output.json}
\end{frame}

\begin{frame}
    \frametitle{Output-Formats: GeoJSON}
    \lstinputlisting{../final-report/code/geojson.c}
\end{frame}

\begin{frame}
    \frametitle{Output-Formats: SQL}
    \lstinputlisting{../final-report/code/output.sql}
\end{frame}

\begin{frame}
    \frametitle{Output-Formats: SQL}
    \lstinputlisting{../final-report/code/sql.c}
\end{frame}

\begin{frame}
    \frametitle{The tool: usage}
    \begin{itemize}
        \item Specifiy weight for city, town, village
        \item Specifiy output format and folder for output
    \end{itemize}
\end{frame}

\begin{frame}
    \frametitle{Demo}
    \begin{center}
        Ludi incipiant
    \end{center}
\end{frame}

\begin{frame}
    \frametitle{Questions?}
    \begin{center}
    Questions?
    \newline Thank you for your attention!
    \end{center}
\end{frame}

\begin{frame}
    \frametitle{Sources}
    \begin{itemize}
        \item foo
    \end{itemize}
\end{frame}

\end{document}
